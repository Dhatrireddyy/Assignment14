\documentclass{beamer}
\usetheme{CambridgeUS}

\title{Assignment 14: Papoulis Chapter 11 }
\author{Dhatri Reddy}
\date{\today}
\logo{\large \LaTeX{}}

\usepackage{amsmath}
\usepackage{romannum}
\usepackage{enumitem}
\setbeamertemplate{caption}[numbered]{}
\providecommand{\pr}[1]{\ensuremath{\Pr\left(#1\right)}}
\providecommand{\cbrak}[1]{\ensuremath{\left\{#1\right\}}}
\providecommand{\brak}[1]{\ensuremath{\left(#1\right)}}


\begin{document}

\begin{frame}
    \titlepage 
\end{frame}

\logo{}

\begin{frame}{Outline}
    \tableofcontents
\end{frame}

\section{Question}
\begin{frame}{Question}
    \begin{block}{Problem 11.13}
    Given a real process x\brak{t} with Fourier transform $X\brak{\omega} = A\brak{\omega} + jB\brak{\omega}$, show that if the processes $A\brak{\omega}$ and $B\brak{\omega}$ satisfy \brak{11-79} and $E\cbrak{A\brak{\omega}} = E\cbrak{B\brak{\omega}} = 0$, then $x\brak{t}$ is WSS.
\end{block}
\end{frame}

\section{Solution}
\begin{frame}
\frametitle{Solution}
$E\cbrak{A\brak{u}A\brak{v}} = Q\brak{u}\delta\brak{u-v} = E\cbrak{B\brak{u}B\brak{v}}$\\
$E\cbrak{A\brak{u}B\brak{v}}=0$\\
for $u\geq0$, $u\geq0$. We have to show that if the above is true and $E\cbrak{A\brak{\omega}} = E\cbrak{B\brak{\omega}} = 0$ then the process

$x\brak{t} = \frac{1}{\pi}\int_{0}^\infty\brak{A\brak{\omega}\cos{\omega t} - B\brak{\omega}\sin{\omega t}}d\omega$ is WSS.

\end{frame}

\section{Proof}
\begin{frame}
\frametitle{Solution}

$E\cbrak{x\brak{t}} = 0$ and 

\begin{align}
     &= \frac{1}{\pi^{2}}\int_{0}^\infty \int_{0}^\infty E\cbrak{A\brak{u}\cos{u\brak{t + \tau}} - B\brak{u}\sin{u\brak{t + \tau}}}\brak{A\brak{v}\cos{vt} - B\brak{v}\sin{vt}}du dv\\
    &= \frac{1}{\pi}\int_{0}^\infty \int_{0}^\infty Q\brak{u}\delta\brak{u-v}\brak{\cos{u\brak{t + \tau}} + \sin{u\brak{t + \tau}}\sin{v\brak{t}}}du dv\\
    &= \frac{1}{\pi^{2}}\int_{0}^\infty Q\brak{u}\cbrak{\cos{u\brak{t + \tau}}\cos{ut} + \sin{u\brak{t + \tau}}\sin{ut}}du\\
    &= \frac{1}{\pi^{2}}\int_{0}^\infty Q\brak{u}\cos{u\tau}du
\end{align}

\end{frame}

\section{Conclusion}
\begin{frame}
\frametitle{Solution}
Hence,\\
From this it follows that x\brak{t} is WSS with $S_{xx} = \frac{Q\brak{\omega}}{\pi}$
\end{frame}
\end{document}